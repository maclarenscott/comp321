\documentclass{article}
\usepackage{graphicx}

\title{COMP 321 -- Draft Problem Statement}
\author{MacLaren Scott}
\date{November 2025}

\begin{document}

\maketitle

\section*{The Evil Letter}

You are writing a very important essay for a notoriously strict professor. Unfortunately, your professor believes that some characters carry ``evil energy'' and wants you to avoid using them too much.

To measure how \emph{evil} a character is, the professor uses its \textbf{ASCII value}. A character with a higher ASCII value is considered more evil. For example, the characters \texttt{a}, \texttt{x}, and \texttt{)} have ASCII values 97, 120, and 41 respectively, so \texttt{x} is the most evil among them.

Your essay is given as a string of characters. The professor defines the \emph{$k$-th most evil character} as the character whose ASCII value is the $k$-th largest among all characters in the essay (counting duplicates separately). Your task is to identify this character.

However, there is a catch: the professor has a terrible memory and uses an old writing tool that cannot store the whole essay in extra memory. You are only allowed to keep track of a \emph{small number of candidate characters at any time}, instead of sorting or copying the entire essay in a separate data structure.

Formally, let $s$ be a string of length $n$. For each character $c$ in $s$, define
\[
f(c) = \text{ASCII value of } c.
\]
Consider the multiset
\[
\{ f(s_1), f(s_2), \dots, f(s_n) \}.
\]
If we sort these values in non-increasing order (from largest to smallest), the $k$-th value in this order corresponds to the $k$-th most evil character. You must output \emph{that} character.

Note that if the same character appears multiple times, each occurrence is counted separately. For example, if $s = \texttt{xxa}$ and $k = 2$, then the ASCII values are
\[
\{120, 120, 97\}
\]
and both the first and second most evil characters are \texttt{x}.

Because of the professor's memory limitations, your solution should use only a small amount of extra space besides the input itself (e.g., keeping track of only $O(k)$ candidate characters at any time).


\section*{Input}

The input consists of two lines:

\begin{itemize}
    \item The first line contains two integers $n$ and $k$ ($1 \le k \le n \le 200{,}000$), the length of the essay and the desired rank of evilness.
    \item The second line contains a string $s$ of length $n$. Each character of $s$ is a visible ASCII character in the range 33 to 126 (from \texttt{!} to \texttt{\~}). There are no spaces in $s$.
\end{itemize}


\section*{Output}

Output a single character: the $k$-th most evil character in the essay, i.e., the character whose ASCII value is the $k$-th largest among all characters in $s$ (counting duplicates separately).


\section*{Constraints}

\begin{itemize}
    \item $1 \le k \le n \le 200{,}000$.
    \item The string $s$ contains only visible ASCII characters in the range 33--126.
    \item Your solution should use only a small amount of extra memory (for example, storing only $O(k)$ candidates rather than the entire string in a secondary array).
\end{itemize}


\section*{Sample Input 1}

\begin{verbatim}
4 2
5364
\end{verbatim}

\section*{Sample Output 1}

\begin{verbatim}
5
\end{verbatim}

\section*{Explanation 1}

The characters are \texttt{5}, \texttt{3}, \texttt{6}, \texttt{4}, with ASCII values
\[
53,\ 51,\ 54,\ 52.
\]
Sorted from largest to smallest, we get
\[
54\ (\texttt{6}),\ 53\ (\texttt{5}),\ 52\ (\texttt{4}),\ 51\ (\texttt{3}).
\]
The 2nd largest value is 53, corresponding to the character \texttt{5}, so the answer is \texttt{5}.


\section*{Sample Input 2}

\begin{verbatim}
3 1
ax)
\end{verbatim}

\section*{Sample Output 2}

\begin{verbatim}
x
\end{verbatim}

\section*{Explanation 2}

The ASCII values are:
\[
\texttt{a} = 97,\quad \texttt{x} = 120,\quad \texttt{)} = 41.
\]
Sorted from largest to smallest, we get
\[
120\ (\texttt{x}),\ 97\ (\texttt{a}),\ 41\ (\texttt{)}).
\]
The 1st most evil character is \texttt{x}, so we output \texttt{x}.

\end{document}
